%Mathematical and Card Counting Topics
\chapter[\cmctzeroshorttitle{}]{\cmctzeroshorttitle{}}

\label{cmct0}

%%%%%%%%%%%%%%%%%%%%%%%%%%%%%%%%%%%%%%%%%%%%%%%%%%%%%%%%%%%%%%%%%%%%%%%%%%%%%%%
%%%%%%%%%%%%%%%%%%%%%%%%%%%%%%%%%%%%%%%%%%%%%%%%%%%%%%%%%%%%%%%%%%%%%%%%%%%%%%%
%%%%%%%%%%%%%%%%%%%%%%%%%%%%%%%%%%%%%%%%%%%%%%%%%%%%%%%%%%%%%%%%%%%%%%%%%%%%%%%
\section{Mathematical Definitions and Topics}
\label{cmct0:smtp0}


%%%%%%%%%%%%%%%%%%%%%%%%%%%%%%%%%%%%%%%%%%%%%%%%%%%%%%%%%%%%%%%%%%%%%%%%%%%%%%%
%%%%%%%%%%%%%%%%%%%%%%%%%%%%%%%%%%%%%%%%%%%%%%%%%%%%%%%%%%%%%%%%%%%%%%%%%%%%%%%
%%%%%%%%%%%%%%%%%%%%%%%%%%%%%%%%%%%%%%%%%%%%%%%%%%%%%%%%%%%%%%%%%%%%%%%%%%%%%%%
\subsection{Use of Greek Letters}
\label{cmct0:smtp0:sugl0}

Greek letters are used as variable names throughout this book, for two
reasons:

\begin{itemize}
   \item There aren't enough letters in the English alphabet.
   \item In some cases, certain quantities are traditionally associated with
         a Greek letter.
\end{itemize}

The Greek alphabet has some upper- and and lower-case letters that appear
very dissimilar:  for example, upper-case ``Omega'' is $\Omega$ whereas
lower-case ``Omega'' is $\omega$.  For this reason, upper- and lower-case
Greek letters are \emph{not} used interchangeably to denote the same variable.


%%%%%%%%%%%%%%%%%%%%%%%%%%%%%%%%%%%%%%%%%%%%%%%%%%%%%%%%%%%%%%%%%%%%%%%%%%%%%%%
%%%%%%%%%%%%%%%%%%%%%%%%%%%%%%%%%%%%%%%%%%%%%%%%%%%%%%%%%%%%%%%%%%%%%%%%%%%%%%%
%%%%%%%%%%%%%%%%%%%%%%%%%%%%%%%%%%%%%%%%%%%%%%%%%%%%%%%%%%%%%%%%%%%%%%%%%%%%%%%
\subsection{Sets of Numbers}
\label{cmct0:smtp0:ssnm0}

An integer is a ``whole number'' or ``counting number''.  
The set of integers ($\{ \ldots, -2, -1, 0, 1, 2, \ldots\}$) is denoted $\vworkintset{}$,
the set of non-negative integers ($\{ 0, 1, 2, \ldots\}$) is denoted $\vworkintsetnonneg{}$,
and the set of positive integers ($\{ 1, 2, \ldots\}$) is denoted $\vworkintsetpos{}$.

A rational number is a ratio (or fraction) of two integers; for example, 2/3 or 121/3 or 0/1.
The set of rational numbers is denoted $\vworkratset{}$, and the set of non-negative rational numbers
is denoted $\vworkratsetnonneg$.

A real number is a number that may have any value, such as 2, -29, 1.8, 2/3, 0, $\pi$, or $\sqrt{2}$.
The set of real numbers is denoted $\vworkrealset$, and the set of non-negative real numbers
is denoted $\vworkrealsetnonneg$.


%%%%%%%%%%%%%%%%%%%%%%%%%%%%%%%%%%%%%%%%%%%%%%%%%%%%%%%%%%%%%%%%%%%%%%%%%%%%%%%
%%%%%%%%%%%%%%%%%%%%%%%%%%%%%%%%%%%%%%%%%%%%%%%%%%%%%%%%%%%%%%%%%%%%%%%%%%%%%%%
%%%%%%%%%%%%%%%%%%%%%%%%%%%%%%%%%%%%%%%%%%%%%%%%%%%%%%%%%%%%%%%%%%%%%%%%%%%%%%%
\subsection{Floor, Ceiling, and Round Functions}
\label{cmct0:smtp0:sfcr0}

The floor of $x$, denoted $\lfloor x \rfloor$, is the largest integer not larger
than $x$.
For example, $\lfloor -7 \rfloor = -7$,
$\lfloor -3.2 \rfloor = -4$,
$\lfloor 3 \rfloor = 3$, and
$\lfloor 4.9 \rfloor = 4$.

The floor function as defined for negative arguments isn't consistent with the
way people tend to think during card counting calculations.  I define
$\lfloor x \rfloor^*$ as a floor function that truncates towards 0 for negative 
numbers.
For example, $\lfloor -7 \rfloor^* = -7$,
$\lfloor -3.2 \rfloor^* = -3$,
$\lfloor 3 \rfloor^* = 3$, and
$\lfloor 4.9 \rfloor^* = 4$.

The ceiling of $x$, denoted $\lceil x \rceil$, is the smallest integer not smaller
than $x$.
For example, $\lceil -7 \rceil = -7$,
$\lceil -3.2 \rceil = -3$,
$\lceil 3 \rceil = 3$, and
$\lceil 4.9 \rceil = 5$.

The ceiling function as defined for negative arguments isn't consistent with the
way people tend to think during card counting calculations.  I define
$\lceil x \rceil^*$ as a ceiling function that rounds away from 0 for negative 
numbers.
For example, $\lceil -7 \rceil^* = -7$,
$\lceil -3.2 \rceil^* = -4$,
$\lceil 3 \rceil^* = 3$, and
$\lceil 4.9 \rceil^* = 5$.

I define $\langle x \rangle$ to be a rounding function where the half-value is rounded towards 0.
For example, $\langle -3.5001 \rangle = -4$,
$\langle -3.5 \rangle = -3$,
$\langle -3.4999 \rangle = -3$,
$\langle 3.4999 \rangle = 3$,
$\langle 3.5 \rangle = 3$,
and $\langle 3.5001 \rangle = 4$.


%%%%%%%%%%%%%%%%%%%%%%%%%%%%%%%%%%%%%%%%%%%%%%%%%%%%%%%%%%%%%%%%%%%%%%%%%%%%%%%
%%%%%%%%%%%%%%%%%%%%%%%%%%%%%%%%%%%%%%%%%%%%%%%%%%%%%%%%%%%%%%%%%%%%%%%%%%%%%%%
%%%%%%%%%%%%%%%%%%%%%%%%%%%%%%%%%%%%%%%%%%%%%%%%%%%%%%%%%%%%%%%%%%%%%%%%%%%%%%%
\subsection{Matrices and Vectors}
\label{cmct0:smtp0:smvc0}

A matrix is a rectangular collection of integers, rational numbers, or real
numbers.  For example,

\begin{equation}
\label{eq:cmct0:smtp0:smvc0:01}
\boldsymbol{m} = \boldsymbol{m_{4 \times 3}} = \left[ \begin{array}{ccc}
1        & \pi & \sqrt{41} \\
1.8      & -9  & 3/2       \\
\pi^\pi  &  2  & 2^{-2}    \\
10       &  0  & 12
\end{array}\right].
\end{equation}

Variable names representing matrices are typeset in a bold font
($\boldsymbol{m}$ rather than $m$).  Optionally, for clarity,
the number of rows and columns in a matrix is included in a subscript,
as is shown in (\ref{eq:cmct0:smtp0:smvc0:01}).

If a single element of a matrix is to be identified, this is done in a subscript,
with the row being identified first, then the column.
For example, from (\ref{eq:cmct0:smtp0:smvc0:01}), $m_{3,1} = \pi^\pi$.
$m_{3,1}$ is not typeset in a bold font, as it is a single element
of a matrix rather than a matrix.

A vector is a matrix with only one row or one column.  If the matrix has only one row,
it is called a row vector, whereas if it has only one column, it is called a columan vector.
Vectors are denoted by a non-bold variable name with an arrow above.
(\ref{eq:cmct0:smtp0:smvc0:02}) is an example of a row vector, and
(\ref{eq:cmct0:smtp0:smvc0:03}) is an example of a column vector.

\begin{equation}
\label{eq:cmct0:smtp0:smvc0:02}
\overrightarrow{p} = \left[ \begin{array}{ccc}
-99        & \pi^2 & \sqrt{3/2} 
\end{array}\right].
\end{equation}

\begin{equation}
\label{eq:cmct0:smtp0:smvc0:03}
\overrightarrow{q} = \left[ \begin{array}{c}
-1        \\
3.29      \\
\pi^3  \\
101
\end{array}\right].
\end{equation}

A single element or a row vector or column vector may be
identified with either one or two subscripts.
From (\ref{eq:cmct0:smtp0:smvc0:02}), $p_{1,3} = p_{3} = \sqrt{3/2}$.
From (\ref{eq:cmct0:smtp0:smvc0:03}), $q_{3,1} = q_{3} = \pi^3$.


%%%%%%%%%%%%%%%%%%%%%%%%%%%%%%%%%%%%%%%%%%%%%%%%%%%%%%%%%%%%%%%%%%%%%%%%%%%%%%%
%%%%%%%%%%%%%%%%%%%%%%%%%%%%%%%%%%%%%%%%%%%%%%%%%%%%%%%%%%%%%%%%%%%%%%%%%%%%%%%
%%%%%%%%%%%%%%%%%%%%%%%%%%%%%%%%%%%%%%%%%%%%%%%%%%%%%%%%%%%%%%%%%%%%%%%%%%%%%%%
\subsection{Matrix Operations}
\label{cmct0:smtp0:smop0}

A matrix with its rows and columns transposed may be identified with a ``T'' subscript.
With reference to (\ref{eq:cmct0:smtp0:smvc0:01}):

\begin{equation}
\label{eq:cmct0:smtp0:smop0:01}
\boldsymbol{m^T} = \boldsymbol{m_{3 \times 4}^T} = \left[ \begin{array}{cccc}
1         & 1.8      & \pi^\pi  & 10 \\
\pi       & -9       & 2        &  0 \\
\sqrt{41} & 3/2      & 2^{-2}   & 12
\end{array}\right].
\end{equation}

A transposed row vector is a column vector, and vice-versa.

Matrices with the same number of rows and columns may be added or subtracted by
adding or subtracting each
element.  For example:

\begin{equation}
\label{eq:cmct0:smtp0:smop0:02}
\left[ \begin{array}{cc}
1 & 2 \\
3 & 4 \\
5 & 6 \\
7 & 8
\end{array}\right]
+
\left[ \begin{array}{cc}
9  &  10 \\
11 &  12 \\
13 &  14 \\
15 &  16
\end{array}\right]
-
\left[ \begin{array}{cc}
17 & 18 \\
19 & 20 \\
21 & 22 \\
23 & 24
\end{array}\right] 
=
\left[ \begin{array}{cc}
-7 & -6 \\
-5 & -4 \\
-3 & -2 \\
-1 & 0
\end{array}\right]
.
\end{equation}

Matrices may be multiplied by a real number by multiplying each element.  For example:

\begin{equation}
\label{eq:cmct0:smtp0:smop0:03}
2 \left[ \begin{array}{cc}
17 & 18 \\
19 & 20 \\
21 & 22 \\
23 & 24
\end{array}\right] 
=
\left[ \begin{array}{cc}
34 & 36 \\
38 & 40 \\
42 & 44 \\
46 & 48
\end{array}\right]
.
\end{equation}

Two matrices $\boldsymbol{A_{m,n}}$ and $\boldsymbol{B_{n,p}}$ may be multiplied
to form the matrix $\boldsymbol{C_{m,p}} = \boldsymbol{A_{m,n}} \boldsymbol{B_{n,p}}$.
Note that it is required that the number of columns of $\boldsymbol{A}$ equal the
number of rows of $\boldsymbol{B}$.  Note also that the matrix multiplication operation
depends on the order of the operands:  $\boldsymbol{A_{m,n}} \boldsymbol{B_{n,p}}$ is
defined, whereas $\boldsymbol{B_{n,p}}\boldsymbol{A_{m,n}}$ ($p \neq m$) is not.

If $\boldsymbol{A_{m,n}}$ and $\boldsymbol{B_{n,p}}$ are multiplied
to form the matrix $\boldsymbol{C_{m,p}} = \boldsymbol{A_{m,n}} \boldsymbol{B_{n,p}}$,
each element of $\boldsymbol{C_{m,p}}$ is defined as:

\begin{equation}
\label{eq:cmct0:smtp0:smop0:04}
c_{i,j} = a_{i,1} b_{1,j} + a_{i,2} b+{2,j} + \ldots +
a_{i,n} b+{n,j} =
\sum_{k=1}^{n} a_{i,k} b_{k,j}
.
\end{equation}

For example,

\begin{eqnarray}
\nonumber & \left[ \begin{array}{ccc}
1 & 2 & 3 \\
4 & 5 & 6
\end{array}\right] 
\left[ \begin{array}{cc}
 7 & 8   \\
 9 & 10  \\
11 & 12
\end{array}\right] & \\
\label{eq:cmct0:smtp0:smop0:05}
& =
\left[ \begin{array}{cc}
(1 \times 7) + (2 \times 9) + (3 \times 11) & (1 \times 8) + (2 \times 10) + (3 \times 12) \\
(4 \times 7) + (5 \times 9) + (6 \times 11) & (4 \times 8) + (5 \times 10) + (6 \times 12)
\end{array}\right] & \\
\nonumber & =
\left[ \begin{array}{cc}
58 & 64 \\
139 & 154
\end{array}\right] .
\end{eqnarray}


%%%%%%%%%%%%%%%%%%%%%%%%%%%%%%%%%%%%%%%%%%%%%%%%%%%%%%%%%%%%%%%%%%%%%%%%%%%%%%%
%%%%%%%%%%%%%%%%%%%%%%%%%%%%%%%%%%%%%%%%%%%%%%%%%%%%%%%%%%%%%%%%%%%%%%%%%%%%%%%
%%%%%%%%%%%%%%%%%%%%%%%%%%%%%%%%%%%%%%%%%%%%%%%%%%%%%%%%%%%%%%%%%%%%%%%%%%%%%%%
\section{Card Counting Definitions and Topics}
\label{cmct0:sctp0}



%%%%%%%%%%%%%%%%%%%%%%%%%%%%%%%%%%%%%%%%%%%%%%%%%%%%%%%%%%%%%%%%%%%%%%%%%%%%%%%
%%%%%%%%%%%%%%%%%%%%%%%%%%%%%%%%%%%%%%%%%%%%%%%%%%%%%%%%%%%%%%%%%%%%%%%%%%%%%%%
%%%%%%%%%%%%%%%%%%%%%%%%%%%%%%%%%%%%%%%%%%%%%%%%%%%%%%%%%%%%%%%%%%%%%%%%%%%%%%%
\subsection{Number of Decks and Cards in the Game}
\label{cmct0:sctp0:nga0}

I use $N_{decks} \in \vworkintsetpos$ to denote the number of decks used in the game.
Most typically, $1 \leq N_{decks} \leq 8$.

I use $N_{cardsperdeck} \in \vworkintsetpos$ to denote the number of cards per deck.
Most typically, $N_{cardsperdeck} = 52$.

I use $N_{cards} \in \vworkintsetpos$ to denote the number of cards used in the game.

\begin{equation}
\label{eq:cmct0:sctp0:nga0:01}
N_{cards} =  N_{decks} N_{cardsperdeck}
\end{equation}

I use $n_{dealt}$ to denote the number of cards removed from the shoe, and $n_{shoe}$ to
denote the number of cards remaining in the shoe.

\begin{equation}
\label{eq:cmct0:sctp0:nga0:02}
n_{dealt} \in \{ 0, \ldots , N_{cards} \}
\end{equation}

\begin{equation}
\label{eq:cmct0:sctp0:nga0:02}
n_{shoe} \in \{ 0, \ldots , N_{cards} \}
\end{equation}

\begin{equation}
\label{eq:cmct0:sctp0:nga0:03}
n_{shoe} = N_{cards} - n_{dealt}
\end{equation}


%%%%%%%%%%%%%%%%%%%%%%%%%%%%%%%%%%%%%%%%%%%%%%%%%%%%%%%%%%%%%%%%%%%%%%%%%%%%%%%
%%%%%%%%%%%%%%%%%%%%%%%%%%%%%%%%%%%%%%%%%%%%%%%%%%%%%%%%%%%%%%%%%%%%%%%%%%%%%%%
%%%%%%%%%%%%%%%%%%%%%%%%%%%%%%%%%%%%%%%%%%%%%%%%%%%%%%%%%%%%%%%%%%%%%%%%%%%%%%%
\subsection{Player's Estimate of $n_{shoe}$}
\label{cmct0:sctp0:npes0}

There are three methods available for a player to obtain or estimate 
$n_{dealt}$ (and hence $n_{shoe})$:

\begin{itemize}
\item Counting the cards as they are removed
      from the shoe.
\item Counting the number of hands played so far from the shoe and multiplying by 
      the number of players (likely including the dealer) and the average number
      of cards per hand.
\item Visually estimating the number of half-decks or decks remaining in the shoe.
\end{itemize}

I use $\hat{n}_{halfdecks\_f}$, $\hat{n}_{halfdeck\_r}$, and 
$\hat{n}_{halfdeck\_c}$ to denote the player's estimate of the number of
half-decks remaining in the shoe using the 3 obvious methods (floor, round, and ceiling).

\begin{equation}
\label{eq:cmct0:sctp0:npes0:03a}
\hat{n}_{halfdecks\_f} = \left\lfloor \frac{2 n_{shoe}}{N_{cardsperdeck}} \right\rfloor
\end{equation}

\begin{equation}
\label{eq:cmct0:sctp0:npes0:03b}
\hat{n}_{halfdecks\_r} = \left\langle \frac{2 n_{shoe}}{N_{cardsperdeck}} \right\rangle
\end{equation}

\begin{equation}
\label{eq:cmct0:sctp0:npes0:03c}
\hat{n}_{halfdecks\_c} = \left\lceil \frac{2 n_{shoe}}{N_{cardsperdeck}} \right\rceil
\end{equation}

Similarly, I use $\hat{n}_{decks\_f}$, $\hat{n}_{decks\_r}$, and $\hat{n}_{decks\_c}$
to denote the player's estimate of the number of
decks remaining in the shoe using the 3 obvious methods (floor, round, and ceiling).

\begin{equation}
\label{eq:cmct0:sctp0:npes0:04a}
\hat{n}_{decks\_f} = \left\lfloor \frac{n_{shoe}}{N_{cardsperdeck}} \right\rfloor
\end{equation}

\begin{equation}
\label{eq:cmct0:sctp0:npes0:04b}
\hat{n}_{decks\_r} = \left\langle \frac{n_{shoe}}{N_{cardsperdeck}} \right\rangle
\end{equation}

\begin{equation}
\label{eq:cmct0:sctp0:npes0:04c}
\hat{n}_{decks\_c} = \left\lceil \frac{n_{shoe}}{N_{cardsperdeck}} \right\rceil
\end{equation}

The advantage of estimating $n_{shoe}$ in half-decks is that true counts
can be calculated as a ratio of integers rather than a ratio of an integer and
a real number.


%%%%%%%%%%%%%%%%%%%%%%%%%%%%%%%%%%%%%%%%%%%%%%%%%%%%%%%%%%%%%%%%%%%%%%%%%%%%%%%
%%%%%%%%%%%%%%%%%%%%%%%%%%%%%%%%%%%%%%%%%%%%%%%%%%%%%%%%%%%%%%%%%%%%%%%%%%%%%%%
%%%%%%%%%%%%%%%%%%%%%%%%%%%%%%%%%%%%%%%%%%%%%%%%%%%%%%%%%%%%%%%%%%%%%%%%%%%%%%%
\subsection{Shoe Penetration}
\label{cmct0:sctp0:nspn0}

I denote shoe penetration by $\rho$.

\begin{equation}
\label{eq:cmct0:sctp0:nspn0:01}
\rho = \frac{n_{dealt}}{N_{cards}} \in \vworkratsetnonneg
\end{equation}

\begin{equation}
\label{eq:cmct0:sctp0:nspn0:02}
0 \leq \rho \leq 1
\end{equation}

Typically, the dealer places a plastic card in the shoe to mark the reshuffle point.
If the plastic card is reached or passed during play, the shoe is reshuffled
when the current hand has concluded.  I use $\rho_{reshuffle}$ to indicate the reshuffle point.
The shoe is reshuffled after the current hand whenever $\rho \geq \rho_{reshuffle}$.


%%%%%%%%%%%%%%%%%%%%%%%%%%%%%%%%%%%%%%%%%%%%%%%%%%%%%%%%%%%%%%%%%%%%%%%%%%%%%%%
%%%%%%%%%%%%%%%%%%%%%%%%%%%%%%%%%%%%%%%%%%%%%%%%%%%%%%%%%%%%%%%%%%%%%%%%%%%%%%%
%%%%%%%%%%%%%%%%%%%%%%%%%%%%%%%%%%%%%%%%%%%%%%%%%%%%%%%%%%%%%%%%%%%%%%%%%%%%%%%
\subsection{Raw Counts}
\label{cmct0:sctp0:swct0}

There are 13 unique values of cards:  2, 3, 4, 5, 6, 7, 8, 9, 10, J, Q, K, and A.
Within each value, there are 4 suits:  clubs, diamonds, hearts, and spades.  I use the letter `T'
to denote a 10-card, and use the letters C, D, H, and S to denote the 4 suits.
Thus, I denote the 52 unique cards as 2C, 2D, 2H, 2S,
3C, 3D, 3H, 3S,
4C, 4D, 4H, 4S,
5C, 5D, 5H, 5S,
6C, 6D, 6H, 6S,
7C, 7D, 7H, 7S,
8C, 8D, 8H, 8S,
9C, 9D, 9H, 9S,
TC, TD, TH, TS,
JC, JD, JH, JS,
QC, QD, QH, QS,
KC, KD, KH, KS,
AC, AD, AH, and AS.


The count of cards that have been dealt so far from the shoe can be represented
as a column vector,

\begin{equation}
\label{eq:cmct0:sctp0:swct0:01}
\overrightarrow{c_{ws}}
=
\left[ \begin{array}{c}
c_{ws2C}   \\
c_{ws2D}   \\
c_{ws2H}   \\
c_{ws2S}   \\
c_{ws3C}   \\
c_{ws3D}   \\
c_{ws3H}   \\
c_{ws3S}   \\
\vdots     \\
c_{wsKC}   \\
c_{wsKD}   \\
c_{wsKH}   \\
c_{wsKS}   \\
c_{wsAC}   \\
c_{wsAD}   \\
c_{wsAH}   \\
c_{wsAS}
\end{array}\right],
\end{equation}

\noindent{}where \emph{w} is used to denote ra\emph{w} count and 
\emph{s} is used to denote \emph{s}uited.

The column vector shown in (\ref{eq:cmct0:sctp0:swct0:01}) might be appropriate
for expressing a strategy for counting against side bets, but Blackjack itself
doesn't consider the suits of cards.  In this case, we can use a simpler count
of cards that have been dealt so far from a shoe,

\begin{equation}
\label{eq:cmct0:sctp0:swct0:02}
\overrightarrow{c_{wu}}
=
\left[ \begin{array}{c}
c_{wu2}   \\
c_{wu3}   \\
c_{wu4}   \\
c_{wu5}   \\
c_{wu6}   \\
c_{wu7}   \\
c_{wu8}   \\
c_{wu9}   \\
c_{wuT}   \\
c_{wuJ}   \\
c_{wuQ}   \\
c_{wuK}   \\
c_{wuA}
\end{array}\right],
\end{equation}

\noindent{}where \emph{w} is used to denote ra\emph{w} count and 
\emph{u} is used to denote \emph{u}nsuited.


%%%%%%%%%%%%%%%%%%%%%%%%%%%%%%%%%%%%%%%%%%%%%%%%%%%%%%%%%%%%%%%%%%%%%%%%%%%%%%%
%%%%%%%%%%%%%%%%%%%%%%%%%%%%%%%%%%%%%%%%%%%%%%%%%%%%%%%%%%%%%%%%%%%%%%%%%%%%%%%
%%%%%%%%%%%%%%%%%%%%%%%%%%%%%%%%%%%%%%%%%%%%%%%%%%%%%%%%%%%%%%%%%%%%%%%%%%%%%%%
\subsection{Running Counts}
\label{cmct0:sctp0:srct0}

Running counts are a row vector, $\overrightarrow{c_{r}}$, that represent the
result of assigning sets of integer values to each card that can be dealt, and keeping
a running tally.  Running counts don't necessarily start at 0 on a new shoe---I denote the
initial row vector $\overrightarrow{c_{r0}}$.

Running counts can be compactly expressed as the sum of $\overrightarrow{c_{r0}}$
and the result of a matrix multiplication
of the raw count and a running count strategy matrix, $\boldsymbol{\psi_{rcs}}$ (suited) or
$\boldsymbol{\psi_{rcu}}$ (unsuited).

\begin{equation}
\label{eq:cmct0:sctp0:srct0:01a}
\overrightarrow{c_{r}}
=
\overrightarrow{c_{r0}} +
\boldsymbol{\psi_{rcs}} \overrightarrow{c_{ws}}
\end{equation}

\begin{equation}
\label{eq:cmct0:sctp0:srct0:01b}
\overrightarrow{c_{r}}
=
\overrightarrow{c_{r0}} +
\boldsymbol{\psi_{rcu}} \overrightarrow{c_{wu}}
\end{equation}


In the case of a simple counting strategy, such as hi-lo, the running count


For example, in the case of a running count strategy that assigns 1 to the card
values 2 through 6 and $-1$ to the face cards and aces, plus keeps a side count of aces,

\begin{equation}
\label{eq:cmct0:sctp0:srct0:02}
\boldsymbol{\psi_{rcu}}
=
\left[ \begin{array}{cc}
 1 & 0     \\
 1 & 0     \\
 1 & 0     \\
 1 & 0     \\
 1 & 0     \\
 0 & 0     \\
 0 & 0     \\
 0 & 0     \\
-1 & 0     \\
-1 & 0     \\
-1 & 0     \\
-1 & 0     \\
-1 & 1
\end{array}\right].
\end{equation}


%%%%%%%%%%%%%%%%%%%%%%%%%%%%%%%%%%%%%%%%%%%%%%%%%%%%%%%%%%%%%%%%%%%%%%%%%%%%%%%
%%%%%%%%%%%%%%%%%%%%%%%%%%%%%%%%%%%%%%%%%%%%%%%%%%%%%%%%%%%%%%%%%%%%%%%%%%%%%%%
%%%%%%%%%%%%%%%%%%%%%%%%%%%%%%%%%%%%%%%%%%%%%%%%%%%%%%%%%%%%%%%%%%%%%%%%%%%%%%%
\subsection{True Counts}
\label{cmct0:sctp0:strc0}

True counts are running counts divided by the some approximation of the 
number of cards remaining
in the shoe, using one of the approximations specified
in (\ref{eq:cmct0:sctp0:npes0:03a}) through (\ref{eq:cmct0:sctp0:npes0:04c}).

Need to find variable names for this.


%%%%%%%%%%%%%%%%%%%%%%%%%%%%%%%%%%%%%%%%%%%%%%%%%%%%%%%%%%%%%%%%%%%%%%%%%%%%%%%
%%%%%%%%%%%%%%%%%%%%%%%%%%%%%%%%%%%%%%%%%%%%%%%%%%%%%%%%%%%%%%%%%%%%%%%%%%%%%%%
%%%%%%%%%%%%%%%%%%%%%%%%%%%%%%%%%%%%%%%%%%%%%%%%%%%%%%%%%%%%%%%%%%%%%%%%%%%%%%%
\subsection{Indexed Playing Strategy}
\label{cmct0:sctp0:sips0}

In general, the indexed playing strategy is a strategy that is some function of
running counts, true counts, estimate of the number of decks, etc.  Again,
need nomenclature.


%%%%%%%%%%%%%%%%%%%%%%%%%%%%%%%%%%%%%%%%%%%%%%%%%%%%%%%%%%%%%%%%%%%%%%%%%%%%%%%
%%%%%%%%%%%%%%%%%%%%%%%%%%%%%%%%%%%%%%%%%%%%%%%%%%%%%%%%%%%%%%%%%%%%%%%%%%%%%%%
%%%%%%%%%%%%%%%%%%%%%%%%%%%%%%%%%%%%%%%%%%%%%%%%%%%%%%%%%%%%%%%%%%%%%%%%%%%%%%%
\subsection{Betting Strategy}
\label{cmct0:sctp0:sibs0}

Need to find some metric that correlates well with the expected value.

Need nomenclature.


%%%%%%%%%%%%%%%%%%%%%%%%%%%%%%%%%%%%%%%%%%%%%%%%%%%%%%%%%%%%%%%%%%%%%%%%%%%%%%%
%%%%%%%%%%%%%%%%%%%%%%%%%%%%%%%%%%%%%%%%%%%%%%%%%%%%%%%%%%%%%%%%%%%%%%%%%%%%%%%
%%%%%%%%%%%%%%%%%%%%%%%%%%%%%%%%%%%%%%%%%%%%%%%%%%%%%%%%%%%%%%%%%%%%%%%%%%%%%%%
\subsection{How to Measure the Goodness of a Card Counting System}
\label{cmct0:sctp0:sgcs0}

Card counting is essentially a partitioning system.  Calculations are performed
to create a set of discrete inputs.  Only some may occur in an actual game.
We decide that some combinations have a positive expected value, and some have a
zero or negative expected value.

$E_i$ denotes the expected profit or loss in the $i$th partition.

If we used only indexed play (didn't adjust betting amount), and played every hand,
the expected value of the game would be.

\begin{equation}
\label{eq:cmct0:sctp0:sgcs0:01}
\sum_{\forall i} p_i E_i
\end{equation}

One measure of goodness of a blackjack card counting system would be the
expected value of those hands played where $E_i > 0$, i.e.

\begin{equation}
\label{eq:cmct0:sctp0:sgcs0:02}
\sum_{i: E_i > 0} p_i E_i
\end{equation}

\noindent{}This measure corresponds to the gain per hand dealt (i.e. per unit time) if one
played only the hands with a favorable count.

Another measure is:

\begin{equation}
\label{eq:cmct0:sctp0:sgcs0:03}
\frac{\sum_{i: E_i > 0} p_i E_i}{\sum_{i: E_i > 0}p_i}
\end{equation}

\noindent{}This measure corresponds to the gain per hand played if one
played only the hands with a favorable count.

A third measure would be the ratio of the good and bad hands, i.e.

\begin{equation}
\label{eq:cmct0:sctp0:sgcs0:04}
-\frac{\sum_{i: E_i \leq 0} p_i E_i}{\sum_{i: E_i > 0}p_i E_i}
\end{equation}

\noindent{}This corresponds to the bet spread required to break even.

One question is which of the three metrics is most useful.  I can't pick out
one as being the most useful.

The first metric may not be especially useful, as
skipping fractions of a shoe may attract heat.

The second metric may have some bearing on wonging strategies if one
is able to hop from table to table so that playing time is not wasted.

The third metric may have some bearing if one plays every hand at the same
table.

Another question is whether a card counting system that does better on one metric
will also do better on the other two.  The answer is not necessarily.
It likely comes down to the effect that indexed play has.
Need to work this down conceptually and symbolically.

