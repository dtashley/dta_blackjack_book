\chapter[\ccchzeroshorttitle{}]{\ccchzerolongtitle{}}

\label{ccch0}

\beginchapterquote{``\ldots{} Beauty is the first test:  there is no permanent
                   place in the world for ugly mathematics.''}
                   {G.H. Hardy \cite{bibref:b:mathematiciansapology:1940},
                   p.85}

\section{Introduction}
%Section Tag: INT


\section{crc32}

\begin{tclcommandname}{crc32}%
generates the CRC-32 of a file or string.  This CRC can be reliably used to obtain
digital signatures of files or data.
\end{tclcommandname}

\begin{tclcommandsynopsis}
\tclcommandsynopsisline{crc32}{filename}
\tclcommandsynopsisline{crc32}{-string binarystringval}
\tclcommandsynopsisline{crc32}{-initialstate}
\tclcommandsynopsisline{crc32}{-advancestate state filename}
\tclcommandsynopsisline{crc32}{-advancestate -string state binarystringval}
\tclcommandsynopsisline{crc32}{-crcfromstate state}
\end{tclcommandsynopsis}

\begin{tclcommanddescription}
The \emph{crc32} command forms the CRC-32 of the binary contents of a file
or of the binary contents of a string.  The CRC-32 is useful as a digital
signature, and can be used with unity probability to determine that two
files are different, or with a probability of about $1-2^{-32}$ to determine
that two files are almost certainly identical.  

In the invocations below, the CRC-32 is always returned as a 10-character ASCII string
of the form \emph{``0xDDDDDDDD''}, where \emph{``DDDDDDDD''} is the hexadecimal representation
of the 32-bit CRC-32, and \emph{``0x''} is a constant 2-character prefix which is included for
aesthetics.  It is guaranteed that:

\begin{itemize}
\item The string returned will be exclusive ASCII.
\item The string will have a length of exactly 10 characters.
\item The first two characters of the string will be \emph{``0x''}.
\item Any letters in the hexadecimal representation will be upper-case.
\end{itemize}

\begin{tclcommandinternaldescription}{\tclcommanddescsynopsisline{crc32}{filename}}
Returns the CRC-32 of \emph{filename}, treated as an ordered collection of bytes (i.e.
newline characters and file termination characters are not treated---the file is 
treated as a binary file).  \emph{filename} must be specified in the form accepted by
the Tcl internals (forward slashes only).
\end{tclcommandinternaldescription}

\begin{tclcommandinternaldescription}{\tclcommanddescsynopsisline{crc32}{-string binarystringval}}
Returns the CRC-32 of \emph{binarystringval}, treated as an ordered collection of bytes (i.e.
newline characters and string termination characters are not honored---the string is 
treated as a binary string).\footnote{For an ASCII string, the \emph{crc32} extension will
behave as expected, and will process all characters up to but not including the zero
terminator.  However, the \emph{crc32} extension will also correctly process non-ASCII strings.}
\end{tclcommandinternaldescription}





\begin{tclcommandinternaldescription}{\tclcommanddescsynopsisline{crc32}{-initialstate}%
                                      \tclcommanddescsynopsisline{crc32}{-advancestate state filename}%
                                      \tclcommanddescsynopsisline{crc32}{-advancestate -string state filename}%
                                      \tclcommanddescsynopsisline{crc32}{-crcfromstate state}%
                                     }
The four forms above are designed to allow ``running CRCs'' to be calculated; in which
the CRC is calculated piecemeal.  These forms allow the caller to retain the internal
state vector of the CRC calculation algorithm.

The first form, \emph{crc32 -initialstate}, returns an ASCII representation of the
correct initial state vector of the CRC-32 state machine.  The client is required
to obtain this initial state before beginning a piecemeal CRC calculation.  Although the
returned string is a constant (it will always be the same), representational details
may change in future versions of the \emph{crc32} extension, and so a caller should never
make assumptions about what this invocation will return, as these assumptions may 
render a script incompatible with future versions of \emph{crc32}.

The second and third forms, \emph{crc32 -advancestate state filename}
and \emph{crc32 -advancestate -string state filename}, apply a file or a binary string
to \emph{state} to produce a new \emph{state}, which is returned.  This new \emph{state}
must be retained by the caller and used in subsequent calls.

The final form, \emph{crc32 -crcfromstate state}, maps from the state vector to the 
calculated CRC, and will return a 10-character ASCII string as described above.
\end{tclcommandinternaldescription}

\end{tclcommanddescription}


\begin{tclcommandusagenotes}
The ``piecemeal'' forms are as efficient as the file and string forms---there is no difference
in the internal algorithms.  The primary cost of the piecemeal forms is in importing and 
exporting the algorithm state vector to/from an ASCII string.  
Thus, the piecemeal forms become less efficient when
small files or strings are processed, as there are more exports and imports
of the state vector.  When using the piecemeal forms, processing the data in 
larger chunks will give better performance.
\end{tclcommandusagenotes}

%End of file c_cch0.tex

